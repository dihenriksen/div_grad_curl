\documentclass{article}
\usepackage{gensymb}
\usepackage{amsmath}
\usepackage[margin=1in]{geometry}
\usepackage{cancel}


\title{Div Grad Curl Solutions}
\author{Dan Henriksen}

\begin{document}
\newcommand{\bb}[1]{\textbf{#1}}
\newcommand{\ii}[1]{\textit{#1}}
% partial derivative:
\newcommand{\pd}[2][f]{\frac{\partial{#1}}{\partial{#2}}}
% unit vector normal to a plane:
\newcommand{\nhat}[2]{\hat{\bb{n}} &= \frac{-\bb{i} \left( {#1} \right) -\bb{j} \left( {#2} \right) + \bb{k}} {\sqrt{1 + ({#1})^2 + ({#2})^2 }}}
\maketitle

This is a work in progress. I am using it to learn LaTeX and vector calculus at once.

\section{Introduction, Vector Functions, and Electrostatics}
\subsection*{Relevant Equations}
\begin{equation}\tag{Coulomb's Law}
    \bb{F} = \frac{1}{4 \pi \epsilon_0}\frac{qq_0}{r^2}\hat{\bb{u}}
\end{equation}
\begin{equation}\tag{Electric Field}
    \bb{E}(\bb{r}) = \frac{\bb{F}(\bb{r})}{q_0} = \frac{1}{4 \pi \epsilon_0}\frac{q}{r^2}\hat{\bb{u}}
\end{equation}

\subsection*{Problems}
\begin{enumerate}
    \item TODO
    \item TODO
    \item
    (a) Write a formula for a vector function
    in two dimensions which is in the positive radial direction
    and whose magnitude is 1.

    Solution:
    \[ \vec{F}(x, y) = \frac{x\textbf{i} + y\textbf{j}}
    {\sqrt{x^2 + y^2}} \]
    The length of a vector is given by
    \[ |\vec{v}| =  \sqrt{x^2 + y^2} \]
    For $\vec{F}$ to be positive in $\vec{r}$, we need
    $\vec{F}(x,y) = x\textbf{i} + y\textbf{j}$, and we divide
    by the length of $\vec{F}$ to get a unit vector in that direction.
    \newline
    \newline
    (b) Write a formula for a vector function in two dimensions
    whose direction makes an angle of 45\degree with the x-axis and whose magnitude
    at any point (x,y) is $(x + y)^2$.

    Solution:

    $\vec{F}$ must be the same in the $\vec{x}$ and $\vec{y}$ directions
    to have a 45\degree angle with the x-axis
    \[ \vec{F}(x, y) = a\textbf{i} + a\textbf{j}\]
    To have a magnitude of $(x + y)^2$, we need:
    \[ |\vec{F}| = \sqrt{a^2 + a^2} = (x + y)^2 \]
    \[ \sqrt{2}a = (x + y)^2 \]
    \[ a = \frac{\sqrt{2}(x + y)^2}{2} \]
    \[ \Rightarrow \vec{F}(x,y) = \frac{\sqrt{2}(x+y)^2}{2}(\textbf{i} + \textbf{j}) \]

    (c) Write a formula for a vector function in two dimensions
    whose direction is tangential [orthogonal to the radial direction] and
    whose magnitude at any point (x,y) is equal to its distance
    from the origin.

    Solution:

    To get the function orthogonal to radial, we need the \textbf{i}
    to depend on -y, and the \textbf{j} to depend on
    x. So
    \[ \vec{F}(x,y) = -y\textbf{i} + x\textbf{j} \]
    The magnitude must be equal to the distance from the origin. The distance
    is given by $d=\sqrt{x^2+y^2}$, so:
    \[ |\vec{F}| = \sqrt{a^2+b^2} = \sqrt{x^2+y^2} \]
    This works with a=y and b=x, with no other changes to the
    magnitude of $\vec{F}(x,y)$, so:
    \[ \vec{F}(x,y) = -y\textbf{i} + x\textbf{j} \]

    (d) Write a formula for a vector function in three
    dimensions which is in the positive radial direction
    and whose magnitude is 1.

    Solution:

    \[ \vec{F}(x,y,z) = x\textbf{i} + y\textbf{j} + z\textbf{k} \]
    To get a unit vector (magnitude = 1), divide this by the length:
    \[ \vec{F}(x,y,z) = \frac{x\textbf{i} + y\textbf{j} + z\textbf{k}}
    {\sqrt{x^2+y^2+z^2}} \]

    \item An object moves in the xy-plane in such a way that its position
    vector \bb{r} is given as a function of time \ii{t} by
    \[ \bb{r} = \bb{i}a\cos{\omega\ii{t}} + \bb{j}b\sin{\omega\ii{t}} \]
    where \ii{a}, \ii{b}, and $\omega$ are constants.

    (a) How far is the object from the origin at any time \ii{t}?

    Solution:
    \begin{align*}
        &d = \sqrt{x^2 + y^2} \\
        &d = \sqrt{a^2\cos^2{(\omega t)} + b^2\sin^2{(\omega t)}}
    \end{align*}
    The important point is that $\cos^2()+\sin^2() = 1$ cannot be factored out.

    (b) Find the object's velocity and acceleration as functions of time.

    Solution:
    \begin{align*}
        \vec{v} &= \frac{dx}{dt} \\
        \vec{v} &= \frac{d}{dt}(\bb{i}a\cos{\omega\ii{t}} + \bb{j}b\sin{\omega\ii{t}}) \\
        \vec{v} &= -\bb{i}a\omega\sin{\omega t} + \bb{j}b\omega\cos{\omega t}
    \end{align*}
    Similarly,
    \begin{align*}
        \vec{a} &= \frac{dv}{dt} \\
        \vec{a} &= \frac{d}{dt}(-\bb{i}a\omega\sin{\omega t} + \bb{j}b\omega\cos{\omega t}) \\
        \vec{a} &= -\bb{i}a\omega^2\cos{\omega t} - \bb{j}b\omega^2\sin{\omega t} \\
    \end{align*}
    Using the definition of \bb{r}, the last line can also be written:
    \[ \vec{a} = -\omega^2\bb{r} \]

    (c) Show that the object moves on the elliptical path
    \[ {\left(\frac{x}{a}\right)}^2 + {\left(\frac{y}{b}\right)}^2 = 1\]

    Solution:

    The x- and y- coordinates of the object are given by:
    \[ x = a\cos{\omega t} \quad  y = b\sin{\omega t} \]
    If we plug these into the left hand side of the ellipse equation, we get:
    \begin{align*}
        {\left(\frac{x}{a}\right)}^2 + {\left(\frac{y}{b}\right)}^2 &= {\left(\frac{\cancel{a^2}\cos^2{\omega t}}{\cancel{a^2}} \right)} + {\left(\frac{\cancel{b^2}\sin^2{\omega t}}{\cancel{b^2}} \right)} \\
        {\left(\frac{x}{a}\right)}^2 + {\left(\frac{y}{b}\right)}^2 &= 1
    \end{align*}
    since $\cos^2() + \sin^2() = 1$.

    \item A charge +1 is situated at the point (1, 0, 0) and a charge -1 is
    situated at the point (-1, 0, 0). Find the electric field of these two charges
    at an arbitrary point (0, y, 0) on the y-axis.

    Solution:

    An electric field is given by:
    \[ \bb{E}(\bb{r}) = \frac{1}{4\pi\epsilon_0}\sum_{l=1}^{N}\frac{q_l}{|\bb{r}-\bb{r}_l|^2} \hat{\bb{u}_l} \]
    We have two charges at (1,0,0) and (-1,0,0). The y-components will cancel out, and the
    x-components will reinforce one another. So we need to find the x-component of each
    charge and add them. For the charge at (1,0,0):
    \begin{align*}
        E_x &= -\frac{1}{4 \pi \epsilon_0} \frac{\cos{\theta}}{r^2} \\
    % TODO - insert image showing coordinates
        \cos{\theta} &= \frac{adj}{hyp} = \frac{x}{r} = \frac{1}{r} \\
        r &= \sqrt{1 + y^2} \\
        \Rightarrow E_x &= -\frac{1}{4 \pi \epsilon_0} \frac{1}{(1+y^2)^\frac{3}{2}} \\
    \end{align*}
    The field from the charge at (-1,0,0) is the same in the x-direction. Adding them together:
    \[ E = -\frac{1}{2 \pi \epsilon_0} \frac{1}{(1+y^2)^\frac{3}{2}} \bb{i} \]

    \item Instead of using arrows to represent vector functions, we sometimes use families
    of curves called \ii{field lines}. A curve y = y(x) is a field line of the vector
    function \bb{F}(x,y) if at each point $(x_0, y_0)$ on the curve $\bb{F}(x_0, y_0)$
    is tangent to the curve (see the figure). % TODO - add figure

    (a) Show that the field lines y = y(x) of a vector function
    \[ \bb{F}(x,y) = \bb{i}F_x(x,y) + \bb{j}F_y(x,y) \]
    are solutions to the differential equation
    \[ \frac{dy}{dx} = \frac{F_y(x,y)}{F_x(x,y)} \]

    Solution:

    The field lines are tangent to the vector function. The tangent is the derivative of
    F, which is given by $\frac{dy}{dx}$. $F_y$ and $F_x$ are the components of F. The
    slope at any point is given by $\frac{F_y}{F_x}$. The slope is the same as the tangent,
    so
    \[ \frac{dy}{dx} = \frac{F_y(x,y)}{F_x(x,y)} \]

    (b) Determine the field lines of each of the functions of Problem I-1. Draw the
    field lines and compare with the arrow diagrams of Problem I-1.

    Solution:

    We can use the relationship derived in (a) to determine the differential equations
    we need to solve to get the field lines.

    (i) $\bb{i}y + \bb{j}x$
    \[ \frac{dy}{dx} = \frac{F_y}{F_x} = \frac{x}{y} \]
    This is a separable differential equation.
    \begin{align*}
        y' &= \frac{x}{y} \\
        ydy &= xdx \\
        \int{ydy} &= \int{xdx} \\
        y^2 &= x^2 + c
    \end{align*}

    (ii) $(\bb{i} + \bb{j})/\sqrt{2}$
    \begin{align*}
        \frac{dy}{dx} &= \frac{\sqrt{2}}{\sqrt{2}} = 1 \\
        \int{dy} &= \int{dx} \\
        y &= x + c
    \end{align*}
    (iii) $\bb{i}x - \bb{j}y$
    \begin{align*}
        \frac{dy}{dx} &= \frac{-y}{x} \\
        \int{\frac{-dy}{y}} &= \int{\frac{dx}{x}} \\
        -\ln{y} &= \ln{x} + c \\
        c &= \ln{x} + \ln{y} = \ln(xy) \\
        c &= xy
    \end{align*}
    (iv) \bb{i}y
    \begin{align*}
        \frac{dy}{dx} &= \frac{0}{y} = 0 \\
        \int{dy} &= 0 \\
        y &= c
    \end{align*}
    (v) \bb{j}x
    \begin{align*}
        \frac{dy}{dx} &= \frac{0}{0} = undefined \\
        \frac{dx}{dy} &= \frac{0}{x} = 0 \\
        x &= c
    \end{align*}
    (vi) $(\bb{i}y + \bb{j}x)/\sqrt{x^2 + y^2}, (x,y)\neq(0,0)$
    \begin{align*}
        \frac{dy}{dx} &= \frac{x}{\sqrt{x^2 + y^2}}\frac{\sqrt{x^2 + y^2}}{y} = \frac{x}{y} \\
        \int{ydy} &= \int{xdx} \\
        y^2 - x^2 &= c
    \end{align*}
    (vii) \bb{i}y + \bb{j}xy
    \begin{align*}
        \frac{dy}{dx} &= \frac{xy}{y} = x \\
        \int{dy} &= \int{xdx} \\
        y &= \frac{x^2}{2} + c
    \end{align*}
    (viii) \bb{i} + \bb{j}y
    \begin{align*}
        \frac{dy}{dx} &= y \\
        \int{\frac{dy}{y}} &= \int{dx} \\
        \ln{y} &= x + c \\
        y &= e^x + c
    \end{align*}

\end{enumerate}

\section{Surface Integrals and the Divergence}
\subsection*{Relevant Equations}
% \begin{align*}
    \begin{equation}\tag{II-4}
        \hat{\bb{n}}(x,y,z) = \frac{\bb{u} \times \bb{v}}{|\bb{u} \times \bb{v}|} = \frac{-\bb{i}\frac{\partial{f}}{\partial{x}} -\bb{j}\frac{\partial{f}}{\partial{y}} + \bb{k}} {\sqrt{1 + (\frac{\partial{f}}{\partial{x}})^2 + (\frac{\partial{f}}{\partial{y}})^2 }}
    \end{equation}
    \begin{equation}\tag{Gauss's Law}
        \int\int_S \bb{E} \cdot \hat{\bb{n}} dS = \frac{q}{\epsilon_0}
    \end{equation}
    \begin{equation}\tag{Gauss's Law Differential Form}
        \textnormal{div} \bb{E} = \nabla \cdot \bb{E} = \frac{\partial{E_x}}{\partial{x}} + \frac{\partial{E_y}}{\partial{y}} + \frac{\partial{E_z}}{\partial{z}} = \frac{\rho}{\epsilon_0}
    \end{equation}
% \end{align*}

\subsection*{Examples}
\begin{itemize}
    \item (Page 25) Solution to double integral
    % TODO - insert graphic
    \[ \sqrt{3}\int\int_R (1-y) dxdy \]
    with $S = z = f(x,y) = 1-x-y$.

    The region R is the triangle in the positive region of the xy-plane
    delimited by the two axes and the line $y = 1-x$.

    Putting the right limits in the double integral we get:
    \begin{align*}
    \sqrt{3}\int\int_R (1-y) dxdy &= \sqrt{3}\int_0^1\int_0^{1-y} (1-y) dxdy \\
    &= \sqrt{3}\int_0^1 (1-y)x \big|_0^{1-y} dy \\
    &= \sqrt{3}\int_0^1 (1-y)^2 dy \\
    &= \sqrt{3}\int_0^1 (1-2y-y^2) dy \\
    &= \sqrt{3}(y-y^2-\frac{y^3}{3}) \big|_0^1 \\
    &= \sqrt{3}(1-1-\frac{1}{3}) \\
    &= \frac{1}{\sqrt{3}}
    \end{align*}

    \item (Page 27) Solution to the double integral
    \[ \int\int_S z^2dS = \int\int_R \sqrt{1-x^2-y^2}dxdy\]
    with $S = x^2+y^2+z^2 = 1$, and R is the projection of S
    in the positive region of the xy-plane.

    R is a quarter circle defined by $x^2+y^2=1$. Converting to
    polar coordinates, R is the region from $r = [0,1]$ and
    $\theta = [0, \frac{\pi}{2}]$.
    \begin{align*}
        \int\int_S z^2dS &= \int\int_R \sqrt{1-x^2-y^2}dxdy \\
        &= \int_0^\frac{\pi}{2}\int_0^1 (\sqrt{1-r^2\cos^2\theta-r^2\sin^2\theta})rdrd\theta \\
        &= \int_0^\frac{\pi}{2}\int_0^1 r\sqrt{1-r^2}drd\theta \\
    \end{align*}
    let $u = \sqrt{1-r^2}$, then
    \[ du = \frac{1}{2\sqrt{1-r^2}}2rdr = \frac{rdr}{\sqrt{1-r^2}} = \frac{rdr}{u} \]
    \[ \Rightarrow rdr = udu \]
    After converting the limits of the definite integral in r to limits in u, we have
    \begin{align*}
        \int\int_S z^2dS &= \int_0^\frac{\pi}{2}\int_1^0 u^2dud\theta \\
        &= \int_0^\frac{\pi}{2} \frac{u^3}{3} \bigg|_{1}^{0} d\theta \\
        &= \frac{1}{3} \int_0^\frac{\pi}{2}d\theta \\
        &= \frac{\pi}{6} \\
    \end{align*}

    \item (Page 29) Solution to the double integral
    \[ \int\int_S \bb{F}\cdot\hat{\bb{n}}dS = \int\int_R \left( \frac{3x}{4} - \frac{3y}{2} + \frac{1}{2} \right) dxdy \]
    where $S = x + 2y + 2z = 2$, and R is the area in the positve region of the xy-plane
    with $y = 1 - x/2$.
    \begin{align*}
        \int\int_S \bb{F}\cdot\hat{\bb{n}}dS &= \int_0^1\int_0^{2-2y} \left( \frac{3x}{4} - \frac{3y}{2} + \frac{1}{2} \right) dxdy \\
        &= \int_0^1\left( \frac{3x^2}{8} - \frac{3yx}{2} + \frac{x}{2} \right) \bigg|_0^{2-2y} dy \\
        &= \int_0^1\left( \frac{3(2-2y)^2}{8} - \frac{3y(2-2y)}{2} + \frac{2-2y}{2} \right) dy \\
        &= \int_0^1\left( \frac{12-24y+12y^2}{8} + \frac{-6y+6y^2}{2} + \frac{2-2y}{2} \right) dy \\
        &= \int_0^1\left( \frac{12-24y+12y^2}{8} + \frac{2-8y+6y^2}{2} \right) dy \\
        &= \int_0^1\left( \frac{6-12y+6y^2}{4} + \frac{4-16y+12y^2}{4} \right) dy \\
        &= \int_0^1\left( \frac{10-28y+18y^2}{4} \right) dy \\
        &= \frac{1}{4} \left( 10y-14y^2+6y^3 \right) \big|_0^1 \\
        &= \frac{1}{2} \\
    \end{align*}
\end{itemize}

\subsection*{Problems}
\begin{enumerate}
    \item Find a unit vector $\hat{n}$ normal to each of the following surfaces.

    (a) $z = f(x,y) = 2 - x - y$

    Solution:

    We can use the result derived in equation II-4:
    \[ \hat{\bb{n}}(x,y,z) = \frac{\bb{u} \times \bb{v}}{|\bb{u} \times \bb{v}|} = \frac{-\bb{i}\frac{\partial{f}}{\partial{x}} -\bb{j}\frac{\partial{f}}{\partial{y}} + \bb{k}} {\sqrt{1 + (\frac{\partial{f}}{\partial{x}})^2 + (\frac{\partial{f}}{\partial{y}})^2 }} \]
    The partial derivatives are:
    \[ \frac{\partial{f}}{\partial{x}} = -1 \quad \textnormal{and} \quad \frac{\partial{f}}{\partial{y}} = -1 \]
    Then:
    \begin{align*}
        \hat{\bb{n}} &= \frac{-\bb{i}(-1) -\bb{j}(-1) + \bb{k}} {\sqrt{1 + (-1)^2 + (-1)^2 }} \\
        &= \frac{\bb{i} + \bb{j} + \bb{k}}{\sqrt{3}}
    \end{align*}

    (b) $z = (x^2 + y^2)^\frac{1}{2}$

    Solution:
    \[ \frac{\partial{f}}{\partial{x}} = \frac{x}{(x^2 + y^2)^\frac{1}{2}} \quad \textnormal{and} \quad \frac{\partial{f}}{\partial{y}} = \frac{y}{(x^2 + y^2)^\frac{1}{2}} \]
    \begin{align*}
        \hat{\bb{n}} &= \frac{-\bb{i}(\frac{x}{{(x^2 + y^2)}^\frac{1}{2}}) -\bb{j}(\frac{y}{(x^2 + y^2)^\frac{1}{2}}) + \bb{k}} {\sqrt{1 + \left( \frac{x}{(x^2 + y^2)^\frac{1}{2}} \right)^2 + \left( \frac{y}{(x^2 + y^2)^\frac{1}{2}} \right)^2 }} \\
        &= \frac{-\bb{i}(\frac{x}{z}) -\bb{j}(\frac{y}{z}) + \bb{k}} {\sqrt{1 + \frac{x^2}{z^2} + \frac{y^2}{z^2} }} \\
        &= \frac{-x\bb{i} -y\bb{j} + z\bb{k}} {z\sqrt{1 + \frac{x^2}{z^2} + \frac{y^2}{z^2} }} \\
        &= \frac{-x\bb{i} -y\bb{j} + z\bb{k}} {\sqrt{z^2 + x^2 + y^2}} \\
        &= \frac{-x\bb{i} -y\bb{j} + z\bb{k}} {\sqrt{z^2 + z^2}} \\
        &= \frac{-x\bb{i} -y\bb{j} + z\bb{k}} {z\sqrt{2}} \\
    \end{align*}

    (c) $z = (1-x^2)^\frac{1}{2}$

    Solution:
    \[ \frac{\partial{f}}{\partial{x}} = \frac{-x}{{(1 - x^2)}^\frac{1}{2}} = \frac{-x}{z} \quad \textnormal{and} \quad \frac{\partial{f}}{\partial{y}} = 0 \]
    \begin{align*}
        \hat{\bb{n}} &= \frac{-\bb{i} \left( \frac{-x}{z} \right) -\bb{j}(0) + \bb{k}} {\sqrt{1 + {\left( \frac{-x}{z} \right)}^2 + {(0)}^2 }} \\
        &= \frac{x\bb{i} + z\bb{k}} {z\sqrt{1 + { \frac{x^2}{z^2} } }} \\
        &= \frac{x\bb{i} + z\bb{k}} {\sqrt{z^2 + x^2 }} \\
        &= \frac{x\bb{i} + z\bb{k}} {\sqrt{((1-x^2)^\frac{1}{2})^2 + x^2 }} \\
        &= \frac{x\bb{i} + z\bb{k}} {\sqrt{1-x^2 + x^2 }} \\
        &= x\bb{i} + z\bb{k}
    \end{align*}

    (d) $z = x^2 + y^2$

    Solution:
    \[ \frac{\partial{f}}{\partial{x}} = 2x \quad \textnormal{and} \quad \frac{\partial{f}}{\partial{y}} = 2y \]
    \begin{align*}
        \hat{\bb{n}}(x,y,z) &= \frac{-\bb{i}2x -\bb{j}2y + \bb{k}} {\sqrt{1 + (2x)^2 + (2y)^2 }} \\
        &= \frac{-\bb{i}2x -\bb{j}2y + \bb{k}} {\sqrt{1 + 4x^2 + 4y^2}} \\
        &= \frac{-\bb{i}2x -\bb{j}2y + \bb{k}} {\sqrt{1 + 4z}} \\
    \end{align*}

    (e) $z = {\left( 1 - \frac{x^2}{a^2} - \frac{y^2}{a^2} \right)}^\frac{1}{2}$

    Solution:
    \[ \frac{\partial{f}}{\partial{x}} = \frac{-x}{a^2z} \quad \textnormal{and} \quad \frac{\partial{f}}{\partial{y}} = \frac{-y}{a^2z} \]
    \begin{align*}
        \hat{\bb{n}}(x,y,z) &= \frac{-\bb{i}\frac{-x}{a^2z} -\bb{j}\frac{-y}{a^2z} + \bb{k}} {\sqrt{1 + (\frac{-x}{a^2z})^2 + (\frac{-y}{a^2z})^2 }} \\
        &= \frac{\bb{i}x + \bb{j}y + \bb{k}a^2z} {a^2z\sqrt{1 + \frac{x^2}{a^4z^2} + \frac{y^2}{a^4z^2} }} \\
        &= \frac{\bb{i}x + \bb{j}y + \bb{k}a^2z} {a\sqrt{a^2z^2 + x^2/a^2 + y^2/a^2 }} \\
        &= \frac{\bb{i}x + \bb{j}y + \bb{k}a^2z} {a\sqrt{a^2z^2 + 1 - 1 + x^2/a^2 + y^2/a^2 }} \\
        &= \frac{\bb{i}x + \bb{j}y + \bb{k}a^2z} {a\sqrt{a^2z^2 + 1 - (1 - x^2/a^2 - y^2/a^2) }} \\
        &= \frac{\bb{i}x + \bb{j}y + \bb{k}a^2z} {a\sqrt{a^2z^2 + 1 - z^2 }} \\
        &= \frac{\bb{i}x + \bb{j}y + \bb{k}a^2z} {a\sqrt{1 + (a^2 - 1)z^2 }} \\
    \end{align*}

    \item (a) Show that the unit vector normal to the plane
    \[ ax + by + cz = d \]
    is given by
    \[ \hat{\bb{n}} = \pm (\bb{i}a +\bb{b} + \bb{c})/(a^2 + b^2 + c^2)^\frac{1}{2}\]

    % Defined \pd partial derivative command; TODO - go back and replace before this.
    Solution:
    \[ z = f(x,y) = d/c - ax/c - by/c \]
    \[ \pd{x} = -a/c \quad and \quad \pd{y} = -b/c \]
    \begin{align*}
        \nhat{-a/c}{-b/c} \\
        &= \frac{\bb{i}(a/c) + \bb{j}(b/c) + \bb{k}}{\sqrt{1 + a^2/c^2 + b^2/c^2}} \\
        &= \frac{\bb{i}a + \bb{j}b + \bb{k}c}{c\sqrt{1 + a^2/c^2 + b^2/c^2}} \\
        &= \frac{\bb{i}a + \bb{j}b + \bb{k}c}{\sqrt{c^2 + a^2 + b^2}} \\
    \end{align*}

    (b) Explain in geometric terms why this expression for $\hat{\bb{n}}$
    is independent of the constant \ii{d}.

    Solution:

    $\hat{\bb{n}}$ being independent of \ii{d} reflects the fact that there are
    infinitely many parallel planes that this vector is the normal unit vector
    to.

    \item Derive the expressions for the unit normal vector for surfaces
    given by $y=g(x,z)$ and by $x=h(y,z)$. Use each to rederive the expression
    for the normal to the plane given in Problem II-2.

    Solution:

    For the surface described by $y=g(x,z)$, we need two vectors \bb{u} and
    \bb{v} tangent to the surface. To get the first vector, hold z constant,
    and slice the surface with a plane parallel to the xy-plane. This plane
    intersects the surface S on a curve. For a point on that curve, the vector
    \bb{u} tangent to the point, the slope $m = \frac{u_y}{u_x}$. The total
    derivative of the surface at this point is
    \[ dg = \pd[g]{x}dx + \pd[g]{z}dz \]
    On the plane we have used to slice through S, z is fixed, so $dz = 0$:
    \[ dg = \pd[g]{x}dx \quad \Rightarrow \quad \frac{dg}{dx} = \pd[g]{x} \]
    The derivative is the slope of the line tanget to the point, so:
    \[ \frac{u_y}{u_x} = \pd[g]{x} \]
    \[ u_y = \left( \pd[g]{x} \right)u_x \]
    Thus
    \[ \bb{u} = \bb{i}u_x + \bb{j}\left( \pd[g]{x} \right)u_x = \left[ \bb{i} + \bb{j}\left( \pd[g]{x} \right) \right]u_x \]

    For the other vector \bb{v}, hold x fixed and allow z to vary so that
    we slice S with a plane parallel to the yz-plane. Using similar reasoning,
    the slope of \bb{v} tangent to the intersecting curve of S and the slicing
    plane is:
    \[ m = \frac{v_y}{v_z} = \pd[g]{z}\]
    \[ \Rightarrow v_y = \left( \pd[g]{z} \right)v_z \]
    Thus
    \[ \bb{v} = \bb{j}\left( \pd[g]{z} \right)v_z + \bb{k}v_z = \left[ \bb{j}\left( \pd[g]{z} \right) + \bb{k}\right]v_z \]

    The unit normal vector is defined as:
    % \[ \hat{\bb{n}}(x,y,z) = \frac{\bb{u} \times \bb{v}}{|\bb{u} \times \bb{v}|} = \frac{-\bb{i}\frac{\partial{f}}{\partial{x}} -\bb{j}\frac{\partial{f}}{\partial{y}} + \bb{k}} {\sqrt{1 + (\frac{\partial{f}}{\partial{x}})^2 + (\frac{\partial{f}}{\partial{y}})^2 }} \]
    \[ \hat{\bb{n}}(x,y,z) = \frac{\bb{u} \times \bb{v}}{|\bb{u} \times \bb{v}|} \]
    Then,
    \begin{align*}
        \bb{u} \times \bb{v} &= \bb{i}({{u_z}v_y - u_y}v_z) + \bb{j}({u_x}v_z - {u_z}v_x) + \bb{k}({u_y}v_x - {u_x}v_y) \\
        \bb{u} \times \bb{v} &= \bb{i}\left({(0)\pd[g]{z}v_z - \pd[g]{x}u_x}v_z \right) + \bb{j}\left({u_x}v_z - (0)(0)\right) + \bb{k}\left(\pd[g]{x}u_x(0) - {u_x}\pd[g]{z}v_z \right) \\
        \bb{u} \times \bb{v} &= -\bb{i}({\pd[g]{x}u_x}v_z) + \bb{j}({u_x}v_z) - \bb{k}({u_x}\pd[g]{z}v_z) \\
        \bb{u} \times \bb{v} &= \left[ -\bb{i}{\pd[g]{x}} + \bb{j} - \bb{k}\pd[g]{z} \right]{u_x}v_z \\
        |\bb{u} \times \bb{v}| &= \sqrt{{u_x^2}v_z^2\left[ \left(\pd[g]{x}\right)^2 + 1 + \left(\pd[g]{z}\right)^2 \right]} \\
    \end{align*}
    \[ \Rightarrow \hat{\bb{n}} = \frac{-\bb{i}{\pd[g]{x}} + \bb{j} - \bb{k}\pd[g]{z}}{\sqrt{1 + \left(\pd[g]{x}\right)^2 + \left(\pd[g]{z}\right)^2 }} \quad \textnormal{for} \quad y=g(x,z) \]

    The derivation for $x = h(y,z)$ is similar, and the result is:
    \[ \hat{\bb{n}} = \frac{-\bb{i}{\pd[h]{y}} + \bb{j} - \bb{k}\pd[h]{z}}{\sqrt{1 + \left(\pd[h]{y}\right)^2 + \left(\pd[h]{z}\right)^2 }} \quad \textnormal{for} \quad x=h(y,z) \]

    \item In each of the following use Equation II-12 to evaluate the surface integral
    $\int\int_SG(x,y,z)dS$

\end{enumerate}

\end{document}