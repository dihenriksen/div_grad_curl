\documentclass{article}
\usepackage{gensymb}
\usepackage{amsmath}
\usepackage[margin=1in]{geometry}
\usepackage{cancel}


\title{Div Grad Curl Solutions}
\author{Dan Henriksen}

\begin{document}
\newcommand{\bb}[1]{\textbf{#1}}
\newcommand{\ii}[1]{\textit{#1}}
% partial derivative:
\newcommand{\pd}[2][f]{\frac{\partial{#1}}{\partial{#2}}}
% unit vector normal to a plane:
\newcommand{\nhat}[2]{\hat{\bb{n}} &= \frac{-\bb{i} \left( {#1} \right) -\bb{j} \left( {#2} \right) + \bb{k}} {\sqrt{1 + ({#1})^2 + ({#2})^2 }}}
% vector
\newcommand{\vvec}[3]{\bb{i}{#1} + \bb{j}{#2} + \bb{k}{#3}}
\maketitle

This is a work in progress. I am using it to learn LaTeX and vector calculus at once.

\section{Introduction, Vector Functions, and Electrostatics}
\subsection*{Relevant Equations}
\begin{equation}\tag{Coulomb's Law}
    \bb{F} = \frac{1}{4 \pi \epsilon_0}\frac{qq_0}{r^2}\hat{\bb{u}}
\end{equation}
\begin{equation}\tag{Electric Field}
    \bb{E}(\bb{r}) = \frac{\bb{F}(\bb{r})}{q_0} = \frac{1}{4 \pi \epsilon_0}\frac{q}{r^2}\hat{\bb{u}}
\end{equation}

\subsection*{Problems}
\begin{enumerate}
    \item TODO
    \item TODO
    \item
    (a) Write a formula for a vector function
    in two dimensions which is in the positive radial direction
    and whose magnitude is 1.

    Solution:
    \[ \vec{F}(x, y) = \frac{x\textbf{i} + y\textbf{j}}
    {\sqrt{x^2 + y^2}} \]
    The length of a vector is given by
    \[ |\vec{v}| =  \sqrt{x^2 + y^2} \]
    For $\vec{F}$ to be positive in $\vec{r}$, we need
    $\vec{F}(x,y) = x\textbf{i} + y\textbf{j}$, and we divide
    by the length of $\vec{F}$ to get a unit vector in that direction.
    \newline
    \newline
    (b) Write a formula for a vector function in two dimensions
    whose direction makes an angle of 45\degree with the x-axis and whose magnitude
    at any point (x,y) is $(x + y)^2$.

    Solution:

    $\vec{F}$ must be the same in the $\vec{x}$ and $\vec{y}$ directions
    to have a 45\degree angle with the x-axis
    \[ \vec{F}(x, y) = a\textbf{i} + a\textbf{j}\]
    To have a magnitude of $(x + y)^2$, we need:
    \[ |\vec{F}| = \sqrt{a^2 + a^2} = (x + y)^2 \]
    \[ \sqrt{2}a = (x + y)^2 \]
    \[ a = \frac{\sqrt{2}(x + y)^2}{2} \]
    \[ \Rightarrow \vec{F}(x,y) = \frac{\sqrt{2}(x+y)^2}{2}(\textbf{i} + \textbf{j}) \]

    (c) Write a formula for a vector function in two dimensions
    whose direction is tangential [orthogonal to the radial direction] and
    whose magnitude at any point (x,y) is equal to its distance
    from the origin.

    Solution:

    To get the function orthogonal to radial, we need the \textbf{i}
    to depend on -y, and the \textbf{j} to depend on
    x. So
    \[ \vec{F}(x,y) = -y\textbf{i} + x\textbf{j} \]
    The magnitude must be equal to the distance from the origin. The distance
    is given by $d=\sqrt{x^2+y^2}$, so:
    \[ |\vec{F}| = \sqrt{a^2+b^2} = \sqrt{x^2+y^2} \]
    This works with a=y and b=x, with no other changes to the
    magnitude of $\vec{F}(x,y)$, so:
    \[ \vec{F}(x,y) = -y\textbf{i} + x\textbf{j} \]

    (d) Write a formula for a vector function in three
    dimensions which is in the positive radial direction
    and whose magnitude is 1.

    Solution:

    \[ \vec{F}(x,y,z) = x\textbf{i} + y\textbf{j} + z\textbf{k} \]
    To get a unit vector (magnitude = 1), divide this by the length:
    \[ \vec{F}(x,y,z) = \frac{x\textbf{i} + y\textbf{j} + z\textbf{k}}
    {\sqrt{x^2+y^2+z^2}} \]

    \item An object moves in the xy-plane in such a way that its position
    vector \bb{r} is given as a function of time \ii{t} by
    \[ \bb{r} = \bb{i}a\cos{\omega\ii{t}} + \bb{j}b\sin{\omega\ii{t}} \]
    where \ii{a}, \ii{b}, and $\omega$ are constants.

    (a) How far is the object from the origin at any time \ii{t}?

    Solution:
    \begin{align*}
        &d = \sqrt{x^2 + y^2} \\
        &d = \sqrt{a^2\cos^2{(\omega t)} + b^2\sin^2{(\omega t)}}
    \end{align*}
    The important point is that $\cos^2()+\sin^2() = 1$ cannot be factored out.

    (b) Find the object's velocity and acceleration as functions of time.

    Solution:
    \begin{align*}
        \vec{v} &= \frac{dx}{dt} \\
        \vec{v} &= \frac{d}{dt}(\bb{i}a\cos{\omega\ii{t}} + \bb{j}b\sin{\omega\ii{t}}) \\
        \vec{v} &= -\bb{i}a\omega\sin{\omega t} + \bb{j}b\omega\cos{\omega t}
    \end{align*}
    Similarly,
    \begin{align*}
        \vec{a} &= \frac{dv}{dt} \\
        \vec{a} &= \frac{d}{dt}(-\bb{i}a\omega\sin{\omega t} + \bb{j}b\omega\cos{\omega t}) \\
        \vec{a} &= -\bb{i}a\omega^2\cos{\omega t} - \bb{j}b\omega^2\sin{\omega t} \\
    \end{align*}
    Using the definition of \bb{r}, the last line can also be written:
    \[ \vec{a} = -\omega^2\bb{r} \]

    (c) Show that the object moves on the elliptical path
    \[ {\left(\frac{x}{a}\right)}^2 + {\left(\frac{y}{b}\right)}^2 = 1\]

    Solution:

    The x- and y- coordinates of the object are given by:
    \[ x = a\cos{\omega t} \quad  y = b\sin{\omega t} \]
    If we plug these into the left hand side of the ellipse equation, we get:
    \begin{align*}
        {\left(\frac{x}{a}\right)}^2 + {\left(\frac{y}{b}\right)}^2 &= {\left(\frac{\cancel{a^2}\cos^2{\omega t}}{\cancel{a^2}} \right)} + {\left(\frac{\cancel{b^2}\sin^2{\omega t}}{\cancel{b^2}} \right)} \\
        {\left(\frac{x}{a}\right)}^2 + {\left(\frac{y}{b}\right)}^2 &= 1
    \end{align*}
    since $\cos^2() + \sin^2() = 1$.

    \item A charge +1 is situated at the point (1, 0, 0) and a charge -1 is
    situated at the point (-1, 0, 0). Find the electric field of these two charges
    at an arbitrary point (0, y, 0) on the y-axis.

    Solution:

    An electric field is given by:
    \[ \bb{E}(\bb{r}) = \frac{1}{4\pi\epsilon_0}\sum_{l=1}^{N}\frac{q_l}{|\bb{r}-\bb{r}_l|^2} \hat{\bb{u}_l} \]
    We have two charges at (1,0,0) and (-1,0,0). The y-components will cancel out, and the
    x-components will reinforce one another. So we need to find the x-component of each
    charge and add them. For the charge at (1,0,0):
    \begin{align*}
        E_x &= -\frac{1}{4 \pi \epsilon_0} \frac{\cos{\theta}}{r^2} \\
    % TODO - insert image showing coordinates
        \cos{\theta} &= \frac{adj}{hyp} = \frac{x}{r} = \frac{1}{r} \\
        r &= \sqrt{1 + y^2} \\
        \Rightarrow E_x &= -\frac{1}{4 \pi \epsilon_0} \frac{1}{(1+y^2)^\frac{3}{2}} \\
    \end{align*}
    The field from the charge at (-1,0,0) is the same in the x-direction. Adding them together:
    \[ E = -\frac{1}{2 \pi \epsilon_0} \frac{1}{(1+y^2)^\frac{3}{2}} \bb{i} \]

    \item Instead of using arrows to represent vector functions, we sometimes use families
    of curves called \ii{field lines}. A curve y = y(x) is a field line of the vector
    function \bb{F}(x,y) if at each point $(x_0, y_0)$ on the curve $\bb{F}(x_0, y_0)$
    is tangent to the curve (see the figure). % TODO - add figure

    (a) Show that the field lines y = y(x) of a vector function
    \[ \bb{F}(x,y) = \bb{i}F_x(x,y) + \bb{j}F_y(x,y) \]
    are solutions to the differential equation
    \[ \frac{dy}{dx} = \frac{F_y(x,y)}{F_x(x,y)} \]

    Solution:

    The field lines are tangent to the vector function. The tangent is the derivative of
    F, which is given by $\frac{dy}{dx}$. $F_y$ and $F_x$ are the components of F. The
    slope at any point is given by $\frac{F_y}{F_x}$. The slope is the same as the tangent,
    so
    \[ \frac{dy}{dx} = \frac{F_y(x,y)}{F_x(x,y)} \]

    (b) Determine the field lines of each of the functions of Problem I-1. Draw the
    field lines and compare with the arrow diagrams of Problem I-1.

    Solution:

    We can use the relationship derived in (a) to determine the differential equations
    we need to solve to get the field lines.

    (i) $\bb{i}y + \bb{j}x$
    \[ \frac{dy}{dx} = \frac{F_y}{F_x} = \frac{x}{y} \]
    This is a separable differential equation.
    \begin{align*}
        y' &= \frac{x}{y} \\
        ydy &= xdx \\
        \int{ydy} &= \int{xdx} \\
        y^2 &= x^2 + c
    \end{align*}

    (ii) $(\bb{i} + \bb{j})/\sqrt{2}$
    \begin{align*}
        \frac{dy}{dx} &= \frac{\sqrt{2}}{\sqrt{2}} = 1 \\
        \int{dy} &= \int{dx} \\
        y &= x + c
    \end{align*}
    (iii) $\bb{i}x - \bb{j}y$
    \begin{align*}
        \frac{dy}{dx} &= \frac{-y}{x} \\
        \int{\frac{-dy}{y}} &= \int{\frac{dx}{x}} \\
        -\ln{y} &= \ln{x} + c \\
        c &= \ln{x} + \ln{y} = \ln(xy) \\
        c &= xy
    \end{align*}
    (iv) \bb{i}y
    \begin{align*}
        \frac{dy}{dx} &= \frac{0}{y} = 0 \\
        \int{dy} &= 0 \\
        y &= c
    \end{align*}
    (v) \bb{j}x
    \begin{align*}
        \frac{dy}{dx} &= \frac{0}{0} = undefined \\
        \frac{dx}{dy} &= \frac{0}{x} = 0 \\
        x &= c
    \end{align*}
    (vi) $(\bb{i}y + \bb{j}x)/\sqrt{x^2 + y^2}, (x,y)\neq(0,0)$
    \begin{align*}
        \frac{dy}{dx} &= \frac{x}{\sqrt{x^2 + y^2}}\frac{\sqrt{x^2 + y^2}}{y} = \frac{x}{y} \\
        \int{ydy} &= \int{xdx} \\
        y^2 - x^2 &= c
    \end{align*}
    (vii) \bb{i}y + \bb{j}xy
    \begin{align*}
        \frac{dy}{dx} &= \frac{xy}{y} = x \\
        \int{dy} &= \int{xdx} \\
        y &= \frac{x^2}{2} + c
    \end{align*}
    (viii) \bb{i} + \bb{j}y
    \begin{align*}
        \frac{dy}{dx} &= y \\
        \int{\frac{dy}{y}} &= \int{dx} \\
        \ln{y} &= x + c \\
        y &= e^x + c
    \end{align*}

\end{enumerate}

\section{Surface Integrals and the Divergence}
\subsection*{Relevant Equations}
\begin{align*}
    &\hat{\bb{n}}(x,y,z) = \frac{\bb{u} \times \bb{v}}{|\bb{u} \times \bb{v}|} = \frac{-\bb{i}\frac{\partial{f}}{\partial{x}} -\bb{j}\frac{\partial{f}}{\partial{y}} + \bb{k}} {\sqrt{1 + (\frac{\partial{f}}{\partial{x}})^2 + (\frac{\partial{f}}{\partial{y}})^2 }} \tag{II-4} \\
    &\iint_S{G(x,y,z)}dS = \int\int_R{G(x,y,f(x,y))} \cdot \sqrt{1 + {\left( \pd{x} \right)}^2 + {\left( \pd{y} \right)}^2}dxdy \tag{II-12} \\
    &\iint_S \bb{F} \cdot \hat{\bb{n}} dS = \int\int_R \left\{ -F_x[x,y,f(x,y)]\pd{x} - F_y[x,y,f(x,y)]\pd{y} + F_z[x,y,f(x,y)] \right\} dxdy \tag{II-13}\label{II-13} \\
    &\iint_S \bb{E} \cdot \hat{\bb{n}} dS = \frac{q}{\epsilon_0} \tag{Gauss's Law}\label{Gauss} \\
    &\textnormal{div} \bb{E} = \nabla \cdot \bb{E} = \frac{\partial{E_x}}{\partial{x}} + \frac{\partial{E_y}}{\partial{y}} + \frac{\partial{E_z}}{\partial{z}} = \frac{\rho}{\epsilon_0} \tag{Gauss's Law Differential Form} \\
\end{align*}

\subsection*{Examples}
\begin{itemize}
    \item (Page 25) Solution to double integral
    % TODO - insert graphic
    \[ \sqrt{3}\int\int_R (1-y) dxdy \]
    with $S = z = f(x,y) = 1-x-y$.

    The region R is the triangle in the positive region of the xy-plane
    delimited by the two axes and the line $y = 1-x$.

    Putting the right limits in the double integral we get:
    \begin{align*}
    \sqrt{3}\int\int_R (1-y) dxdy &= \sqrt{3}\int_0^1\int_0^{1-y} (1-y) dxdy \\
    &= \sqrt{3}\int_0^1 (1-y)x \big|_0^{1-y} dy \\
    &= \sqrt{3}\int_0^1 (1-y)^2 dy \\
    &= \sqrt{3}\int_0^1 (1-2y-y^2) dy \\
    &= \sqrt{3}(y-y^2-\frac{y^3}{3}) \big|_0^1 \\
    &= \sqrt{3}(1-1-\frac{1}{3}) \\
    &= \frac{1}{\sqrt{3}}
    \end{align*}

    \item (Page 27) Solution to the double integral
    \[ \int\int_S z^2dS = \int\int_R \sqrt{1-x^2-y^2}dxdy\]
    with $S = x^2+y^2+z^2 = 1$, and R is the projection of S
    in the positive region of the xy-plane.

    R is a quarter circle defined by $x^2+y^2=1$. Converting to
    polar coordinates, R is the region from $r = [0,1]$ and
    $\theta = [0, \frac{\pi}{2}]$.
    \begin{align*}
        \int\int_S z^2dS &= \int\int_R \sqrt{1-x^2-y^2}dxdy \\
        &= \int_0^\frac{\pi}{2}\int_0^1 (\sqrt{1-r^2\cos^2\theta-r^2\sin^2\theta})rdrd\theta \\
        &= \int_0^\frac{\pi}{2}\int_0^1 r\sqrt{1-r^2}drd\theta \\
    \end{align*}
    let $u = \sqrt{1-r^2}$, then
    \[ du = \frac{1}{2\sqrt{1-r^2}}2rdr = \frac{rdr}{\sqrt{1-r^2}} = \frac{rdr}{u} \]
    \[ \Rightarrow rdr = udu \]
    After converting the limits of the definite integral in r to limits in u, we have
    \begin{align*}
        \int\int_S z^2dS &= \int_0^\frac{\pi}{2}\int_1^0 u^2dud\theta \\
        &= \int_0^\frac{\pi}{2} \frac{u^3}{3} \bigg|_{1}^{0} d\theta \\
        &= \frac{1}{3} \int_0^\frac{\pi}{2}d\theta \\
        &= \frac{\pi}{6} \\
    \end{align*}

    \item (Page 29) Solution to the double integral
    \[ \int\int_S \bb{F}\cdot\hat{\bb{n}}dS = \int\int_R \left( \frac{3x}{4} - \frac{3y}{2} + \frac{1}{2} \right) dxdy \]
    where $S = x + 2y + 2z = 2$, and R is the area in the positve region of the xy-plane
    with $y = 1 - x/2$.
    \begin{align*}
        \int\int_S \bb{F}\cdot\hat{\bb{n}}dS &= \int_0^1\int_0^{2-2y} \left( \frac{3x}{4} - \frac{3y}{2} + \frac{1}{2} \right) dxdy \\
        &= \int_0^1\left( \frac{3x^2}{8} - \frac{3yx}{2} + \frac{x}{2} \right) \bigg|_0^{2-2y} dy \\
        &= \int_0^1\left( \frac{3(2-2y)^2}{8} - \frac{3y(2-2y)}{2} + \frac{2-2y}{2} \right) dy \\
        &= \int_0^1\left( \frac{12-24y+12y^2}{8} + \frac{-6y+6y^2}{2} + \frac{2-2y}{2} \right) dy \\
        &= \int_0^1\left( \frac{12-24y+12y^2}{8} + \frac{2-8y+6y^2}{2} \right) dy \\
        &= \int_0^1\left( \frac{6-12y+6y^2}{4} + \frac{4-16y+12y^2}{4} \right) dy \\
        &= \int_0^1\left( \frac{10-28y+18y^2}{4} \right) dy \\
        &= \frac{1}{4} \left( 10y-14y^2+6y^3 \right) \big|_0^1 \\
        &= \frac{1}{2} \\
    \end{align*}
\end{itemize}

\subsection*{Problems}
\begin{enumerate}
    \item Find a unit vector $\hat{n}$ normal to each of the following surfaces.

    (a) $z = f(x,y) = 2 - x - y$

    Solution:

    We can use the result derived in equation II-4:
    \[ \hat{\bb{n}}(x,y,z) = \frac{\bb{u} \times \bb{v}}{|\bb{u} \times \bb{v}|} = \frac{-\bb{i}\frac{\partial{f}}{\partial{x}} -\bb{j}\frac{\partial{f}}{\partial{y}} + \bb{k}} {\sqrt{1 + (\frac{\partial{f}}{\partial{x}})^2 + (\frac{\partial{f}}{\partial{y}})^2 }} \]
    The partial derivatives are:
    \[ \frac{\partial{f}}{\partial{x}} = -1 \quad \textnormal{and} \quad \frac{\partial{f}}{\partial{y}} = -1 \]
    Then:
    \begin{align*}
        \hat{\bb{n}} &= \frac{-\bb{i}(-1) -\bb{j}(-1) + \bb{k}} {\sqrt{1 + (-1)^2 + (-1)^2 }} \\
        &= \frac{\bb{i} + \bb{j} + \bb{k}}{\sqrt{3}}
    \end{align*}

    (b) $z = (x^2 + y^2)^\frac{1}{2}$

    Solution:
    \[ \frac{\partial{f}}{\partial{x}} = \frac{x}{(x^2 + y^2)^\frac{1}{2}} \quad \textnormal{and} \quad \frac{\partial{f}}{\partial{y}} = \frac{y}{(x^2 + y^2)^\frac{1}{2}} \]
    \begin{align*}
        \hat{\bb{n}} &= \frac{-\bb{i}(\frac{x}{{(x^2 + y^2)}^\frac{1}{2}}) -\bb{j}(\frac{y}{(x^2 + y^2)^\frac{1}{2}}) + \bb{k}} {\sqrt{1 + \left( \frac{x}{(x^2 + y^2)^\frac{1}{2}} \right)^2 + \left( \frac{y}{(x^2 + y^2)^\frac{1}{2}} \right)^2 }} \\
        &= \frac{-\bb{i}(\frac{x}{z}) -\bb{j}(\frac{y}{z}) + \bb{k}} {\sqrt{1 + \frac{x^2}{z^2} + \frac{y^2}{z^2} }} \\
        &= \frac{-x\bb{i} -y\bb{j} + z\bb{k}} {z\sqrt{1 + \frac{x^2}{z^2} + \frac{y^2}{z^2} }} \\
        &= \frac{-x\bb{i} -y\bb{j} + z\bb{k}} {\sqrt{z^2 + x^2 + y^2}} \\
        &= \frac{-x\bb{i} -y\bb{j} + z\bb{k}} {\sqrt{z^2 + z^2}} \\
        &= \frac{-x\bb{i} -y\bb{j} + z\bb{k}} {z\sqrt{2}} \\
    \end{align*}

    (c) $z = (1-x^2)^\frac{1}{2}$

    Solution:
    \[ \frac{\partial{f}}{\partial{x}} = \frac{-x}{{(1 - x^2)}^\frac{1}{2}} = \frac{-x}{z} \quad \textnormal{and} \quad \frac{\partial{f}}{\partial{y}} = 0 \]
    \begin{align*}
        \hat{\bb{n}} &= \frac{-\bb{i} \left( \frac{-x}{z} \right) -\bb{j}(0) + \bb{k}} {\sqrt{1 + {\left( \frac{-x}{z} \right)}^2 + {(0)}^2 }} \\
        &= \frac{x\bb{i} + z\bb{k}} {z\sqrt{1 + { \frac{x^2}{z^2} } }} \\
        &= \frac{x\bb{i} + z\bb{k}} {\sqrt{z^2 + x^2 }} \\
        &= \frac{x\bb{i} + z\bb{k}} {\sqrt{((1-x^2)^\frac{1}{2})^2 + x^2 }} \\
        &= \frac{x\bb{i} + z\bb{k}} {\sqrt{1-x^2 + x^2 }} \\
        &= x\bb{i} + z\bb{k}
    \end{align*}

    (d) $z = x^2 + y^2$

    Solution:
    \[ \frac{\partial{f}}{\partial{x}} = 2x \quad \textnormal{and} \quad \frac{\partial{f}}{\partial{y}} = 2y \]
    \begin{align*}
        \hat{\bb{n}}(x,y,z) &= \frac{-\bb{i}2x -\bb{j}2y + \bb{k}} {\sqrt{1 + (2x)^2 + (2y)^2 }} \\
        &= \frac{-\bb{i}2x -\bb{j}2y + \bb{k}} {\sqrt{1 + 4x^2 + 4y^2}} \\
        &= \frac{-\bb{i}2x -\bb{j}2y + \bb{k}} {\sqrt{1 + 4z}} \\
    \end{align*}

    (e) $z = {\left( 1 - \frac{x^2}{a^2} - \frac{y^2}{a^2} \right)}^\frac{1}{2}$

    Solution:
    \[ \frac{\partial{f}}{\partial{x}} = \frac{-x}{a^2z} \quad \textnormal{and} \quad \frac{\partial{f}}{\partial{y}} = \frac{-y}{a^2z} \]
    \begin{align*}
        \hat{\bb{n}}(x,y,z) &= \frac{-\bb{i}\frac{-x}{a^2z} -\bb{j}\frac{-y}{a^2z} + \bb{k}} {\sqrt{1 + (\frac{-x}{a^2z})^2 + (\frac{-y}{a^2z})^2 }} \\
        &= \frac{\bb{i}x + \bb{j}y + \bb{k}a^2z} {a^2z\sqrt{1 + \frac{x^2}{a^4z^2} + \frac{y^2}{a^4z^2} }} \\
        &= \frac{\bb{i}x + \bb{j}y + \bb{k}a^2z} {a\sqrt{a^2z^2 + x^2/a^2 + y^2/a^2 }} \\
        &= \frac{\bb{i}x + \bb{j}y + \bb{k}a^2z} {a\sqrt{a^2z^2 + 1 - 1 + x^2/a^2 + y^2/a^2 }} \\
        &= \frac{\bb{i}x + \bb{j}y + \bb{k}a^2z} {a\sqrt{a^2z^2 + 1 - (1 - x^2/a^2 - y^2/a^2) }} \\
        &= \frac{\bb{i}x + \bb{j}y + \bb{k}a^2z} {a\sqrt{a^2z^2 + 1 - z^2 }} \\
        &= \frac{\bb{i}x + \bb{j}y + \bb{k}a^2z} {a\sqrt{1 + (a^2 - 1)z^2 }} \\
    \end{align*}

    \item (a) Show that the unit vector normal to the plane
    \[ ax + by + cz = d \]
    is given by
    \[ \hat{\bb{n}} = \pm (\bb{i}a +\bb{b} + \bb{c})/(a^2 + b^2 + c^2)^\frac{1}{2}\]

    % Defined \pd partial derivative command; TODO - go back and replace before this.
    Solution:
    \[ z = f(x,y) = d/c - ax/c - by/c \]
    \[ \pd{x} = -a/c \quad and \quad \pd{y} = -b/c \]
    \begin{align*}
        \nhat{-a/c}{-b/c} \\
        &= \frac{\bb{i}(a/c) + \bb{j}(b/c) + \bb{k}}{\sqrt{1 + a^2/c^2 + b^2/c^2}} \\
        &= \frac{\bb{i}a + \bb{j}b + \bb{k}c}{c\sqrt{1 + a^2/c^2 + b^2/c^2}} \\
        &= \frac{\bb{i}a + \bb{j}b + \bb{k}c}{\sqrt{c^2 + a^2 + b^2}} \\
    \end{align*}

    (b) Explain in geometric terms why this expression for $\hat{\bb{n}}$
    is independent of the constant \ii{d}.

    Solution:

    $\hat{\bb{n}}$ being independent of \ii{d} reflects the fact that there are
    infinitely many parallel planes that this vector is the normal unit vector
    to.

    \item Derive the expressions for the unit normal vector for surfaces
    given by $y=g(x,z)$ and by $x=h(y,z)$. Use each to rederive the expression
    for the normal to the plane given in Problem II-2.

    Solution:

    For the surface described by $y=g(x,z)$, we need two vectors \bb{u} and
    \bb{v} tangent to the surface. To get the first vector, hold z constant,
    and slice the surface with a plane parallel to the xy-plane. This plane
    intersects the surface S on a curve. For a point on that curve, the vector
    \bb{u} tangent to the point, the slope $m = \frac{u_y}{u_x}$. The total
    derivative of the surface at this point is
    \[ dg = \pd[g]{x}dx + \pd[g]{z}dz \]
    On the plane we have used to slice through S, z is fixed, so $dz = 0$:
    \[ dg = \pd[g]{x}dx \quad \Rightarrow \quad \frac{dg}{dx} = \pd[g]{x} \]
    The derivative is the slope of the line tanget to the point, so:
    \[ \frac{u_y}{u_x} = \pd[g]{x} \]
    \[ u_y = \left( \pd[g]{x} \right)u_x \]
    Thus
    \[ \bb{u} = \bb{i}u_x + \bb{j}\left( \pd[g]{x} \right)u_x = \left[ \bb{i} + \bb{j}\left( \pd[g]{x} \right) \right]u_x \]

    For the other vector \bb{v}, hold x fixed and allow z to vary so that
    we slice S with a plane parallel to the yz-plane. Using similar reasoning,
    the slope of \bb{v} tangent to the intersecting curve of S and the slicing
    plane is:
    \[ m = \frac{v_y}{v_z} = \pd[g]{z}\]
    \[ \Rightarrow v_y = \left( \pd[g]{z} \right)v_z \]
    Thus
    \[ \bb{v} = \bb{j}\left( \pd[g]{z} \right)v_z + \bb{k}v_z = \left[ \bb{j}\left( \pd[g]{z} \right) + \bb{k}\right]v_z \]

    The unit normal vector is defined as:
    % \[ \hat{\bb{n}}(x,y,z) = \frac{\bb{u} \times \bb{v}}{|\bb{u} \times \bb{v}|} = \frac{-\bb{i}\frac{\partial{f}}{\partial{x}} -\bb{j}\frac{\partial{f}}{\partial{y}} + \bb{k}} {\sqrt{1 + (\frac{\partial{f}}{\partial{x}})^2 + (\frac{\partial{f}}{\partial{y}})^2 }} \]
    \[ \hat{\bb{n}}(x,y,z) = \frac{\bb{u} \times \bb{v}}{|\bb{u} \times \bb{v}|} \]
    Then,
    \begin{align*}
        \bb{u} \times \bb{v} &= \bb{i}({{u_z}v_y - u_y}v_z) + \bb{j}({u_x}v_z - {u_z}v_x) + \bb{k}({u_y}v_x - {u_x}v_y) \\
        \bb{u} \times \bb{v} &= \bb{i}\left({(0)\pd[g]{z}v_z - \pd[g]{x}u_x}v_z \right) + \bb{j}\left({u_x}v_z - (0)(0)\right) + \bb{k}\left(\pd[g]{x}u_x(0) - {u_x}\pd[g]{z}v_z \right) \\
        \bb{u} \times \bb{v} &= -\bb{i}({\pd[g]{x}u_x}v_z) + \bb{j}({u_x}v_z) - \bb{k}({u_x}\pd[g]{z}v_z) \\
        \bb{u} \times \bb{v} &= \left[ -\bb{i}{\pd[g]{x}} + \bb{j} - \bb{k}\pd[g]{z} \right]{u_x}v_z \\
        |\bb{u} \times \bb{v}| &= \sqrt{{u_x^2}v_z^2\left[ \left(\pd[g]{x}\right)^2 + 1 + \left(\pd[g]{z}\right)^2 \right]} \\
    \end{align*}
    \[ \Rightarrow \hat{\bb{n}} = \frac{-\bb{i}{\pd[g]{x}} + \bb{j} - \bb{k}\pd[g]{z}}{\sqrt{1 + \left(\pd[g]{x}\right)^2 + \left(\pd[g]{z}\right)^2 }} \quad \textnormal{for} \quad y=g(x,z) \]

    The derivation for $x = h(y,z)$ is similar, and the result is:
    \[ \hat{\bb{n}} = \frac{-\bb{i}{\pd[h]{y}} + \bb{j} - \bb{k}\pd[h]{z}}{\sqrt{1 + \left(\pd[h]{y}\right)^2 + \left(\pd[h]{z}\right)^2 }} \quad \textnormal{for} \quad x=h(y,z) \]

    \item In each of the following use Equation II-12 to evaluate the surface integral
    $\int\int_S{G(x,y,z)}dS$

    (a) $G(x,y,z) = z$, where S is the portion of the plane $x + y + z = 1$ in the first
    octant.

    Solution:

    \[ z = f(x,y) = 1-x-y \]
    \[ \pd{x} = -1 \quad and \quad \pd{y} = -1 \]
    \[ \int\int_S{G(x,y,f(x,y))} \cdot \sqrt{1 + {\left( \pd{x} \right)}^2 + {\left( \pd{y} \right)}^2}dxdy = \int\int_R{z} \cdot \sqrt{1 + (-1)^2 + (-1)^2}dxdy \]
    \[ = \int\int_R{(1 - x - y)}\sqrt{3}dxdy \]
    Where the integral is still expressed merely over R. To get the limits of integration, we must find
    $f(x,y)$ in the first octant, which is when $x \ge 0,  y \ge 0, z \ge 0$. R is the projection of S
    onto the xy-plane, so z = 0.
    \[ 0 \le 1-x-y \]
    \[ x \le 1-y \]
    So the limits of integration for x are $x \in (0,1-y)$. For y, the lower limit is 0 since y must
    be positive. The maximum for y is when x = 0:
    \[ 0 = 1 - 0 - y \]
    \[ y = 1 \]
    Thus $y \in (0,1)$. The integral then is:
    \begin{align*}
        &= \int_0^1\int_0^{1-y}{(1 - x - y)}\sqrt{3}dxdy \\
        &= \sqrt{3}\int_0^1{(x - x^2/2 - yx)}|_0^{1-y}dy \\
        &= \sqrt{3}\int_0^1{(1-y - (1-y)^2/2 - y(1-y))}|_0^{1-y}dy \\
        &= \frac{\sqrt{3}}{2}\int_0^1{(2-2y - (1-2y+y^2) - 2y + 2y^2)}dy \\
        &= \frac{\sqrt{3}}{2}\int_0^1{(1 - 2y + y^2)}dy \\
        &= \frac{\sqrt{3}}{2}{(y - y^2 + y^3/3)}|_0^1 \\
        &= \frac{\sqrt{3}}{2}{(1 - (1)^2 + (1)^3/3)} \\
        &= \frac{\sqrt{3}}{6} \\
    \end{align*}

    (b) $G(x,y,z) = \frac{1}{1 + 4(x^2 + y^2)}$ where S is the portion of the paraboloid $z = x^2 + y^2$ between z=0 and z=1.
    \[ \int\int_S{G(x,y,f(x,y))} \cdot \sqrt{1 + {\left( \pd{x} \right)}^2 + {\left( \pd{y} \right)}^2}dxdy = \int\int_R{\frac{1}{1 + 4(x^2 + y^2)}} \cdot \sqrt{1 + \left(\pd{x}\right)^2 + \left(\pd{y}\right)^2}dxdy \]
    \[ \pd{x} = \pd[]{x}{(x^2 + y^2)} = 2x \]
    \[ \pd{y} = \pd[]{y}{(x^2 + y^2)} = 2y \]
    The region R is the projection of S onto the xy-plane. Since the surface is a paraboloid around
    the z-axis, R is a circle centered at the origin. Using z=1,
    \[ 1 = x^2 + y^2 \]
    This is a circle of radius 1. We can convert to polar coordinates, with $r \in (0,1)$ and
    $\theta \in (0,2\pi)$. $dxdy = rdrd\theta$.
    \begin{align*}
        &= \int\int_R{\frac{1}{1 + 4(x^2 + y^2)}} \cdot \sqrt{1 + \left(\pd{x}\right)^2 + \left(\pd{y}\right)^2}dxdy \\
        &= \int\int_R{\frac{1}{1 + 4(x^2 + y^2)}} \cdot \sqrt{1 + (2x)^2 + (2y)^2}dxdy \\
        &= \int\int_R{\frac{1}{1 + 4(x^2 + y^2)}} \cdot \sqrt{1 + 4(x^2 + y^2)}dxdy \\
        &= \int\int_R{\frac{1}{\sqrt{1 + 4(x^2 + y^2)}}} dxdy \\
        &= \int_0^{2\pi}\int_0^1{\frac{1}{\sqrt{1 + 4r^2}}} rdrd\theta \\
        & \qquad \textnormal{let} \: u = 1 + 4r^2 \\
        & \qquad du = 8rdr \\
        & \qquad r=0 \rightarrow u=1 \\
        & \qquad r=1 \rightarrow u=5 \\
        &= \frac{1}{8}\int_0^{2\pi}\int_1^5{\frac{1}{\sqrt{u}}} dud\theta \\
        &= \frac{1}{8}\int_0^{2\pi}\int_1^5{\frac{1}{\sqrt{u}}} dud\theta \\
        &= \frac{1}{8}\int_0^{2\pi}{ 2u^{1/2} } \big{|}_1^5 d\theta \\
        &= \frac{1}{8}\int_0^{2\pi}{ 2\sqrt{5} - 2 } d\theta \\
        &= \frac{1}{8}{2\pi}\left({ 2\sqrt{5} - 2 }\right) \\
        &= \frac{\pi}{2}\left({ \sqrt{5} - 1 }\right) \\
    \end{align*}

    (c) $G(x,y,z) = (1 - x^2 - y^2)^\frac{3}{2}$, where S is the hemisphere $z = (1 - x^2 - y^2)^\frac{1}{2}$.
    \[ \pd{x} = \pd[]{x}{(1 - x^2 - y^2)^\frac{1}{2}} = \frac{-2x}{2\sqrt{1-x^2-y^2}} = \frac{-x}{z} \]
    \[ \pd{y} = \pd[]{y}{(1 - x^2 - y^2)^\frac{1}{2}} = \frac{-2y}{2\sqrt{1-x^2-y^2}} = \frac{-y}{z} \]
    \[ \int\int_S{G(x,y,f(x,y))} \cdot \sqrt{1 + {\left( \pd{x} \right)}^2 + {\left( \pd{y} \right)}^2}dxdy = \int\int_R{(1 - x^2 - y^2)^\frac{3}{2}} \cdot \sqrt{1 + \left(\pd{x}\right)^2 + \left(\pd{y}\right)^2}dxdy \]
    \begin{align*}
        &= \int\int_R{(z^2)^\frac{3}{2}} \cdot \sqrt{1 + (-x/z)^2 + (-y/z)^2} dxdy \\
        &= \int\int_R{z^3} \sqrt{1 + x^2/z^2 + y^2/z^2} dxdy \\
        &= \int\int_R\frac{z^3}{z} \sqrt{z^2 + x^2 + y^2} dxdy \\
        &= \int\int_R z^2 \sqrt{z^2 + x^2 + y^2} dxdy \\
        &= \int\int_R z^2 \sqrt{(1 - x^2 - y^2) + x^2 + y^2} dxdy \\
        &= \int\int_R z^2 \sqrt{1} dxdy \\
        &= \int\int_R 1 - x^2 - y^2 dxdy \\
    \end{align*}
    The region R is the projection of the hemisphere onto the xy-plane, which is a circle
    centered at the origin with radius 1. Converting to polar coordinates:
    \begin{align*}
        &= \int_0^{2\pi}\int_0^1 (1 - r^2) rdrd\theta \\
        & \quad \textnormal{let} \: u = 1-r^2 \\
        & \quad du = -2rdr \\
        & \quad r=0 \rightarrow u=1 \\
        & \quad r=1 \rightarrow u=0 \\
        &= -\frac{1}{2}\int_0^{2\pi}\int_1^0 (u) dud\theta \\
        &= -\frac{1}{4}\int_0^{2\pi} u^2 |_1^0 d\theta \\
        &= \frac{1}{4}\int_0^{2\pi} d\theta \\
        &= \frac{2\pi}{4} \\
        &= \frac{\pi}{2} \\
    \end{align*}

    \item In each of the following use Equation II-13 to evaluate the surface integral
    $\int\int_S \bb{F} \cdot \bb{n} dS$

    (a) $\bb{F}(x,y,z) = \bb{i}x - \bb{k}z$, where S is the portion of the plane
    x + y + 2z = 2 in the first octant.

    Solution:

    Equation II-13 is:
    \[ \int\int_S \bb{F} \cdot \hat{\bb{n}} dS = \int\int_R \left\{ -F_x[x,y,f(x,y)]\pd{x} - F_y[x,y,f(x,y)]\pd{y} + F_z[x,y,f(x,y)] \right\} dxdy \]
    Thus:
    \[ F_x = x \quad F_y = 0 \quad F_z = -z \]
    \[ z = f(x,y) = 1 - x/2 - y/2 \]
    \[ \pd{x} = -1/2 \quad \pd{y} = -1/2 \]
    R is the projection of S onto the xy-plane, where z=0, so R is bounded by
    \[ x + y + 2(0) = 2 \]
    \[ x = 2-y \]
    \[ \Rightarrow x \in (0, 2-y) \quad y \in (0, 2) \]
    \begin{align*}
        \int\int_S \bb{F} \cdot \hat{\bb{n}} dS &= \int_0^2\int_0^{2-y} \left\{ -(x)(-1/2) - (0) + (-z) \right\} dxdy \\
        &= \int_0^2\int_0^{2-y} (x/2 - 1 + x/2 + y/2) dxdy \\
        &= \int_0^2\int_0^{2-y} (-1 + x + y/2) dxdy \\
        &= 2\int_0^2\int_0^{2-y} (-2 + 2x + y) dxdy \\
        &= 2\int_0^2 (-2x + x^2 + yx) \big{|}_0^{2-y} dy \\
        &= 2\int_0^2 (-2(2-y) + (2-y)^2 + y(2-y)) dy \\
        &= 2\int_0^2 (-4 +2y + 4 - 4y + y^2 + 2y-y^2) dy \\
        &= 2\int_0^2 (0) dy \\
        &= 0 \\
    \end{align*}

    (b) $ \bb{F}(x,y,z) = \bb{i}x + \bb{j}y + \bb{k}z $, where S is the hemisphere
    $ z = \sqrt{a^2 - x^2 - y^2}$.

    Solution:
    \[ F_x = x \quad F_y = y \quad F_z = z \]
    \[ \pd{x} = \frac{-2x}{2\sqrt{a^2 - x^2 - y^2}} = \frac{-x}{z} \quad \pd{y} = \frac{-y}{z} \]
    R is:
    \[ 0 = \sqrt{a^2 - x^2 - y^2} \]
    \[ x^2 + y^2 = a^2 \]
    This is a circle in the xy-plane with radius a.
    \begin{align*}
        \int\int_S \bb{F} \cdot \hat{\bb{n}} dS &= \int\int_R \left\{ -F_x[x,y,f(x,y)]\pd{x} - F_y[x,y,f(x,y)]\pd{y} + F_z[x,y,f(x,y)] \right\} dxdy \\
        &= \int\int_R \left\{ -(x)(-x/z) - (y)(-y/z) + z \right\} dxdy \\
        &= \int\int_R \left\{ x^2/z + y^2/z + z \right\} dxdy \\
        &= \int\int_R (1/z)\left( x^2 + y^2 + z^2 \right) dxdy \\
        &= \int\int_R \frac{a^2}{\sqrt{a^2 - (x^2 + y^2)}} dxdy \\
    \end{align*}
    Converting to polar coordinates:
    \begin{align*}
        &= \int_0^{2\pi}\int_0^a \frac{a^2}{\sqrt{a^2 - r^2}} rdrd\theta \\
        & \quad u = a^2-r^2 \\
        & \quad du = -2rdr \\
        & \quad r=0 \rightarrow u=a^2 \\
        & \quad r=a \rightarrow u=0 \\
        &= \frac{-a^2}{2}\int_0^{2\pi}\int_{a^2}^0 \frac{1}{\sqrt{u}} dud\theta \\
        &= \frac{-a^2}{2}\int_0^{2\pi} 2u^{1/2} \big{|}_{a^2}^0 d\theta \\
        &= -a^2\int_0^{2\pi} -(a^2)^{1/2} d\theta \\
        &= a^3\int_0^{2\pi} d\theta \\
        &= 2{\pi}a^3 \\
    \end{align*}

    (c) $\bb{F}(x,y,z) = \bb{j}y + \bb{k}$, where S is the portion of the paraboloid
    $z = 1 - x^2 - y^2$ above the xy-plane.

    Solution:

    \[ F_x = 0 \quad F_y = y \quad F_z = 1 \]
    \[ \pd{x} = -2x \quad \pd{y} = -2y \]
    R is:
    \[ 0 = 1 - x^2 - y^2 \]
    \[ x^2 + y^2 = 1 \]
    R is a circle on the xy-plane with radius 1.
    \begin{align*}
        \int\int_S \bb{F} \cdot \hat{\bb{n}} dS &= \int\int_R \left\{ -F_x[x,y,f(x,y)]\pd{x} - F_y[x,y,f(x,y)]\pd{y} + F_z[x,y,f(x,y)] \right\} dxdy \\
        &= \int\int_R \left\{ -(0)(-2x) - (y)(-2y) + 1 \right\} dxdy \\
        &= \int\int_R \left( 2y^2 + 1 \right) dxdy \\
        &= \int\int_R 1 dxdy + \int\int_R 2y^2 dxdy \\
    \end{align*}
    In polar coordinates, $y = r\sin{\theta}$:
    \begin{align*}
        &= \int_0^{2\pi}\int_0^1 rdrd\theta + \int_0^{2\pi}\int_0^1 2r^2\sin^2{\theta} rdrd\theta \\
        &= \pi + 2\int_0^{2\pi}\int_0^1 r^3\sin^2{\theta} drd\theta \\
        &= \pi + 2\int_0^{2\pi} \frac{r^4}{4}\big{|}_0^1 \sin^2{\theta} d\theta \\
        &= \pi + (1/2)\int_0^{2\pi} \sin^2{\theta} d\theta \\
        &= \pi + (1/2)\int_0^{2\pi} \left\{ (1/2) - (1/2)\cos{2\theta} \right\} d\theta \\
        &= \pi + (1/2)\int_0^{2\pi} (1/2) d\theta - (1/2)\int_0^{2\pi}(1/2)\cos{2\theta} d\theta \\
        &= \pi + (1/2)(2\pi/2) - (1/2)(1/4)\sin{2\theta} \big{|}_0^{2\pi} \\
        &= \pi + (\pi/2) - (1/8)(\sin{4\pi} - \sin{0}) \\
        &= 3\pi/2
    \end{align*}

    \item The distribution of mass on the hemispherical shell $z = (R^2 - x^2 - y^2)^{1/2}$
    is given by $\sigma(x,y,z) = (\sigma_0/R^2)(x^2 + y^2)$ where $\sigma_0$ is a constant. Find
    an expression in terms of $\sigma_0$ and R for the total mass of the shell.
    \newcommand\z{(R^2 - x^2 - y^2)^{1/2}}
    \newcommand\sig{(\sigma_0/R^2)(x^2 + y^2)}

    Solution:

    To get the total mass of a shell from the mass density, integrate the density over the surface
    of the shell:
    \[ M = \iint_S \sigma(x,y,z) dS \]
    The means of solving this in the spirit of the book is to evaluate the integral of $\sigma$ over the projection of S onto the xy-plane.
    This result is derived in the book, and is:
    \[ M = \iint_S \sigma(x,y,z) dS = \iint_R \sigma(x,y,z) \cdot \sqrt{1 + \pd{x}^2 + \pd{y}^2} dxdy \tag{II-12} \]
    \[ f(x,y) = z(x,y,z) \quad \pd{x} = \frac{-2x}{2\sqrt{R^2 - x^2 - y^2}} = \frac{-x}{z} \quad \pd{y} = {\frac{-y}{z}} \]
    \newcommand\px{\frac{-x}{z}}
    \newcommand\py{\frac{-y}{z}}
    \begin{align*}
        M &= \iint_S \sigma(x,y,z) dS = \iint_R \frac{\sigma_0}{R^2}(x^2 + y^2) \cdot \sqrt{1 + \left( \px \right)^2 + \left( \py \right)^2} dxdy \\
        &= \frac{\sigma_0}{R^2} \iint_R (x^2 + y^2) \cdot \frac{1}{z} \sqrt{z^2 + x^2 + y^2} dxdy \\
        &= \frac{\sigma_0}{R^2} \iint_R (x^2 + y^2) \cdot \frac{1}{\z} \sqrt{({\z})^2 + x^2 + y^2} dxdy \\
        &= \frac{\sigma_0}{R^2} \iint_R (x^2 + y^2) \cdot \frac{R}{\z} dxdy \\
    \end{align*}
    At this point we can convert to polar coordinates, which will also help define the region R
    (different from the constant R).
    \[ r^2 = x^2 + y^2 \quad dxdy = rdrd\theta \]
    \[ R: r \in (0,R), \theta \in (0,2\pi) \]
    \[ M = \frac{\sigma_0}{R} \int_0^{2\pi}\int_0^R \frac{r^3}{\sqrt{R^2 - r^2}} drd\theta \]
    We can solve this through integration by parts:
    \[ \int udv = uv - \int vdu \]
    \[ u = r^2 \quad du = 2rdr \]
    \[ dv = \frac{-2rdr}{(R^2 - r^2)^{1/2}} \quad v = (R^2 - r^2)^{1/2}\]
    (admittedly, this takes some jiggery pokery to figure out.)
    \begin{align*}
        M &= \frac{2\pi\sigma_0}{R} \int_0^R \frac{r^3}{\sqrt{R^2 - r^2}} dr \\
        &= \frac{2\pi\sigma_0}{R} \left( (r^2)(R^2 - r^2)^{1/2} \bigg{|}_0^R - \int_0^R 2r{\sqrt{R^2 - r^2}} dr \right) \\
        &= \frac{2\pi\sigma_0}{R} \left( (R^2)(R^2 - R^2)^{1/2} -(0)(R^2) - \int_0^R 2r{\sqrt{R^2 - r^2}} dr \right) \\
        &= -\frac{2\pi\sigma_0}{R} \int_0^R 2r{\sqrt{R^2 - r^2}} dr \\
        & let \quad u = (R^2 - r^2) \quad du = -2rdr \\
        &= \frac{2\pi\sigma_0}{R} \int_{R^2}^0 {\sqrt{u}} du \\
        &= \frac{2\pi\sigma_0}{R} \frac{2u^{3/2}}{3} \bigg{|}_{R^2}^{0} \\
        &= \frac{4\pi\sigma_0}{3R} (R^2)^{3/2} \\
        &= \frac{4\pi\sigma_0R^2}{3} \\
    \end{align*}

    % TODO - solve this problem in spherical coordinates directly
    % However, this problem can also be solved in spherical coordinates directly, without projecting
    % the surface \ii{z} into the xy-plane.
    % \[ M = \iint_S \frac{\sigma_0}{R^2}(x^2 + y^2) dS \]
    % The hemispherical shell S is the region where:
    % \begin{gather*}
    %     r = R \\
    %     \theta \in (0,2\pi) \\
    %     \phi \in (0,\pi/2) \\
    %     \textnormal{In spherical coordinates:} \\
    %     x = r\sin{\theta}\cos{\phi} \\
    %     y = r\sin{\theta}\sin{\phi} \\
    %     z = r\cos{\theta}
    % \end{gather*}
    % \begin{align*}
    %     M &= \frac{\sigma_0}{R^2} \int_0^{\pi/2}\int_0^{2\pi} (R^2 - z^2)R^2 \sin{\theta} d\theta d\phi \\
    %     &= \sigma_0 \int_0^{\pi/2}\int_0^{2\pi} (R^2 - R^2\cos^2{\theta}) \sin{\theta} d\theta d\phi \\
    %     &= \sigma_0 R^2 \int_0^{\pi/2}\int_0^{2\pi} (1 - \cos^2{\theta}) \sin{\theta} d\theta d\phi \\
    %     &= \sigma_0 R^2 \pi/2 \int_0^{2\pi} (\sin{\theta} - \cos^2{\theta}\sin{\theta}) d\theta \\
    %     &= \frac{\sigma_0 R^2 \pi}{2} (-\cos{\theta}) \big{|}_0^{2\pi} \int_0^{2\pi} (- \cos^2{\theta}\sin{\theta}) d\theta \\
    %     \\
    %     M &= \frac{\sigma_0}{R^2}\int_0^{\pi/2}\int_0^{2\pi} (r^2 \sin^2{\theta}\cos^2{\phi} + r^2 \sin^2{\theta} \sin^2{\phi}) r^2\sin{\theta} d\theta d\phi \bigg{|}_{r=R}\\
    %     &= \frac{\sigma_0}{R^2}\int_0^{\pi/2}\int_0^{2\pi} R^4(\sin^2{\theta}) \sin{\theta} d\theta d\phi \\
    %     &= \sigma_0R^2 \int_0^{\pi/2}\int_0^{2\pi} \sin^3{\theta} d\theta d\phi \\
    %     &= \pi\sigma_0R^2/2 \int_0^{2\pi} \sin^3{\theta} d\theta \\
    % \end{align*}

    \item Find the moment of inertia about the z-axis of the hemispherical
    shell of Problem II-6.

    Solution:

    In general:
    \[ I = \int_0^M r^2 dm \]
    In this case, the mass distribution function is over a surface:
    \[ I = \iint_S \sigma_0(x,y,z)r^2 dS \]
    \[ I = \iint_S \sigma_0(x,y,z)(x^2 + y^2) dS \]
    \[ I = \iint_S \sig(x^2 + y^2) dS \]
    This is solved similarly to the integral in II-6. The surface S is the same,
    so the partial derivatives and R are the same as well:

    \[ I = \frac{\sigma_0}{R^2} \iint_R (x^2 + y^2)^2 \cdot \sqrt{1 + \left(\pd{x}\right)^2 + \left(\pd{y}\right)^2 } dxdy \]
    \[ = \frac{\sigma_0}{R^2} \iint_R (x^2 + y^2)^2 \cdot \sqrt{1 + \left(\px\right)^2 + \left(\py\right)^2 } dxdy \]
    \[ = \frac{\sigma_0}{R^2} \iint_R (x^2 + y^2)^2 \cdot \frac{R}{\z} dxdy \]
    \[ = \frac{\sigma_0}{R^2} \int_0^{2\pi} \int_0^R r^4 \cdot \frac{R}{(R^2 - r^2)^{1/2}} rdr d\theta \]
    \[ = \frac{\sigma_0}{R} \int_0^{2\pi} \int_0^R \frac{r^5}{(R^2 - r^2)^{1/2}} dr d\theta \]
    \[  \qquad \textnormal{let} \quad u = r^4 \quad du = 4r^3dr \]
    \[  \qquad dv = \frac{-2rdr}{2(R^2 - r^2)^{1/2}} = \frac{-rdr}{(R^2 - r^2)^{1/2}} \quad v = (R^2 - r^2)^{1/2} \]
    \[ = \frac{-\sigma_0}{R} \int_0^{2\pi} \int_0^R {r^4}\frac{-r}{(R^2 - r^2)^{1/2}} dr d\theta \]
    \[ = \frac{-\sigma_0}{R} \left[ \int_0^{2\pi} (r^4)(R^2-r^2)^{1/2} \bigg{|}_0^R d\theta - \int_0^{2\pi} \int_0^R {4r^3}(R^2 - r^2)^{1/2} dr d\theta \right] \]
    \[ = \frac{\sigma_0}{R} \int_0^{2\pi} \int_0^R {4r^3}(R^2 - r^2)^{1/2} dr d\theta \]
    \[  \qquad \textnormal{let} \quad u = 4r^2 \quad du = 8rdr \]
    \[  \qquad dv = (-3r)(R^2 - r^2)^{1/2} \quad v = (R^2 - r^2)^{3/2} \]
    \[ = \frac{\sigma_0}{R} \int_0^{2\pi} \int_0^R {4r^2} \left( \frac{-3r}{-3} \right) (R^2 - r^2)^{1/2} dr d\theta \]
    \[ = \frac{-\sigma_0}{3R} \left[ \int_0^{2\pi} (4r^2)(R^2 - r^2)^{3/2} \bigg{|}_0^R d\theta - \int_0^{2\pi} \int_0^R 8r(R^2 - r^2)^{3/2} dr d\theta \right] \]
    \[ = \frac{8\sigma_0}{3R} \int_0^{2\pi} \int_0^R r(R^2 - r^2)^{3/2} dr d\theta \]
    \[  \qquad \textnormal{let} \quad u = R^2 - r^2 \quad du = -2rdr \]
    \[  \qquad r=0 \rightarrow u=R^2 \quad r=R \rightarrow u=0 \]
    \[ = \frac{8\sigma_0}{3R} \int_0^{2\pi} \int_0^R \frac{-2r}{-2}(R^2 - r^2)^{3/2} dr d\theta \]
    \[ = \frac{-4\sigma_0}{3R} \int_0^{2\pi} \int_{R^2}^0 (u)^{3/2} du d\theta \]
    \[ = \frac{-4\sigma_0}{3R} \int_0^{2\pi} (2/5)(u)^{5/2} \bigg{|}_{R^2}^0 d\theta \]
    \[ = \frac{-8\sigma_0}{15R} \int_0^{2\pi} \left[ (0)^{5/2} - (R^2)^{5/2} \right] d\theta \]
    \[ = \frac{8\sigma_0}{15R} \int_0^{2\pi} R^5 d\theta \]
    \[ = \frac{8\sigma_0 R^4}{15} 2 \pi \]
    \[ = \frac{16\pi \sigma_0 R^4}{15} \]

    \item An electrostatic field is given by $ \bb{E} = \lambda (\vvec{yz}{xz}{xy})$,
    where $\lambda$ is a constant. Use Gauss' law to find the total charge enclosed by the surface
    shown in the figure consisting of $S_1$, the hemisphere $ z = (R^2 - x^2 - y^2)^{1/2} $,
    and $S_2$, its circular base in the xy-plane.

    Solution:

    \ref{Gauss} is
    \[ \iint_S \bb{E} \cdot \hat{\bb{n}} dS = \frac{q}{\epsilon_0} \]

    We have to evaluate Gauss's Law over the two surfaces separately. For the base
    $\hat{\bb{n}} = \hat{\bb{k}}$ since the normal vector will point up (or down). Thus
    \begin{gather*}
        \bb{E} \cdot \hat{\bb{n}} = E_z = \lambda xy \\
        \frac{q}{\epsilon_0} = \lambda \iint_R (xy) dS \\
        \quad \textnormal{Converting to polar coordinates} \\
        \quad x = r \cos \theta \quad y = r \sin \theta \\
        \quad \theta \in (0,2\pi) \quad r \in (0,R) \\
        \frac{q}{\epsilon_0} = \lambda \int_0^{2\pi}\int_0^R (r^2 \cos \theta \sin \theta) rdrd\theta \\
    \end{gather*}
    The integral $ \int_0^{2\pi} \sin \theta \cos \theta d\theta$ is well known to be zero, since it is an odd function
    being integrated over a symmetric area. In an odd function, $f(-x) = -f(x)$, so integrating over
    a symmetric area results in zero. Thus for the base,
    \[ \frac{q}{\epsilon_0} = 0 \implies q = 0\]

    For the second part of the surface, the hemisphere:
    \[ f = z(x,y) = (R^2 - x^2 - y^2)^{1/2} \]
    \[ \pd{x} = (-x/z) \quad \pd{y} = (-y/z) \]

    We can use~\ref{II-13} to evaluate this since \bb{E} is not always directed radially:
    \begin{align*}
        \iint_S \bb{E} \cdot \hat{\bb{n}} dS &= \iint_R \left\{ -E_x[x,y,f(x,y)]\pd{x} - E_y[x,y,f(x,y)]\pd{y} + E_z[x,y,f(x,y)] \right\} dxdy \\
        \frac{q}{\epsilon_0} &= \lambda \iint_R \left\{ -(yz)(-x/z) - (xz)(-y/z) + xy \right\} dxdy \\
        &= \lambda \iint_R (yx + xy + xy) dxdy \\
        &= \lambda \iint_R (3xy) dxdy \\
        & \qquad \textnormal{Converting to polar coordinates} \\
        & \qquad x = r \cos \theta \quad y = r \sin \theta \\
        & \qquad \theta \in (0,2\pi) \quad r \in (0,R) \\
        &= 3\lambda \int_0^{2\pi}\int_0^R (r^2 \sin \theta \cos \theta) rdrd\theta \\
    \end{align*}
    This is the same integral as we had for the base, so the answer is zero. Thus in total:
    \[ q = 0 \]

    \item An electrostatic field is given by $\bb{E} = \lambda (\vvec{x}{y}{0})$, where $\lambda$ is a constant. Use Gauss' Law to find the total
    charge enclosed by the surface shown in the figure consisting of $S_1$, the curved portion of the half-cylinder $z = (r^2 - y^2)^{1/2}$ of length h;
    $S_2$ and $S_3$, the two semicircular plane end pieces; and $S_4$, the rectangular portion of the xy-plane. Express your results in terms of
    $\lambda$, r, and h.
    % TODO - insert image

    Solution:

    We have to apply Gauss' Law to each of the four surfaces independently. However, by inspection, the two ends of the half-cylinder will yield the same
    result. Gauss's Law is:
    \[ \iint_S \bb{E} \cdot \hat{\bb{n}} dS = \frac{q}{\epsilon_0} \]

    We will take $S_4$, the long base on the xy-plane, first. The normal vector to this surface is $\hat{\bb{n}} = \hat{\bb{k}}$. Thus:
    \[ \bb{E} \cdot \hat{\bb{n}} = \bb{E} \cdot \hat{\bb{k}} = E_z = 0 \]
    \[ \implies \iint_{S_4} \bb{E} \cdot \hat{\bb{n}} dS = \frac{q}{\epsilon_0} = 0 \]

    Next we take $S_2$, the half-circle end to the cylinder, and will double the result to account for $S_3$.
    The normal vector is $\hat{\bb{n}} = \hat{\bb{i}}$. Thus:
    \[ \bb{E} \cdot \hat{\bb{n}} = \bb{E} \cdot \hat{\bb{i}} = E_x = \lambda h/2 \]
    \[ \implies \iint_{S_2} \bb{E} \cdot \hat{\bb{n}} dS = \frac{q}{\epsilon_0} = \lambda \iint_{S_2} \frac{h}{2} dS \]
    \[ \qquad \textnormal{Converting to polar coordinates:} \]
    \[ \qquad \theta \in (0,\pi) \quad \rho \in (0,r) \]
    \[ \qquad dS = \rho d\rho d\theta \]
    \[ \frac{q}{\epsilon_0} = \lambda \frac{h}{2} \iint_{S_2} dS = \lambda \frac{h}{2} \int_0^\pi \int_0^r \rho d\rho d\theta \]
    where we have use $\rho$ as the polar radial coordinate because r in this case is a constant.
    \[ \frac{q}{\epsilon_0} = \lambda \frac{h}{2}\frac{r^2 \pi}{2} \]
    \[ \implies q = \frac{\lambda h r^2 \pi \epsilon_0}{4} \]
    for both $S_2$ and $S_3$.

    Lastly, $S_1$, the half-cylindrical shell.
    \[ f = z(x,y) = (r^2 - y^2)^{1/2} \]
    \[ \pd{x} = (0) \quad \pd{y} = (-y/z) \]
    Applying~\ref{II-13} to~\ref{Gauss}:
    \[ \iint_S \bb{E} \cdot \hat{\bb{n}} dS = \iint_R \left\{ -E_x[x,y,f(x,y)]\pd{x} - E_y[x,y,f(x,y)]\pd{y} + E_z[x,y,f(x,y)] \right\} dxdy \]
    \[ \frac{q}{\epsilon_0} = \lambda \iint_R \left\{ -x(0) - y(-y/z) + (0) \right\} dxdy \]
    \[ = \lambda \int_{-h/2}^{h/2} \int_{-r}^{r} \frac{y^2}{(r^2 - y^2)^{1/2}} dxdy \]
    \[ = \lambda \left[ x \big{|}_{-h/2}^{h/2} \right] \int_{-r}^{r} \frac{y^2}{(r^2 - y^2)^{1/2}} dy \]
    \[ = h \lambda \int_{-r}^{r} \frac{y^2}{(r^2 - y^2)^{1/2}} dy \]
    \centerline{TODO - work out this integral through trig substitution}
    \[ \frac{q}{\epsilon_0} = h \lambda \frac{\pi r^2}{2} \]
    \[ \implies q = h \lambda \frac{\pi r^2}{2} \epsilon_0 \]

    Adding together the contributions from each surface:
    \[ q = q_{S_1} + q_{S_2} + q_{S_3} + q_{S_4}\]
    \[ q = 0 + \frac{\lambda h r^2 \pi \epsilon_0}{4} + \frac{\lambda h r^2 \pi \epsilon_0}{4} + h \lambda \frac{\pi r^2}{2} \epsilon_0 \]
    \[ q = \lambda h r^2 \pi \epsilon_0 \]
\end{enumerate}

\end{document}